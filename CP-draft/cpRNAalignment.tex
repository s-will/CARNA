\documentclass{llncs}

\usepackage{amsmath}
\usepackage{amssymb}
\usepackage{multicol}

\newcommand{\SET}[1]{\{#1\}}
\newcommand*{\PSET}[2]{\{\,#1\mid#2\,\}}

\newcommand{\weight}{\operatorname{weight}}
\newcommand{\argmax}{\operatorname*{arg max}}
\newcommand{\var}{\texttt}
\newcommand{\cA}{{\mathcal A}}

% \addtolength{\hoffset}{-0.6cm}
% \addtolength{\textwidth}{1.2cm}
% \addtolength{\voffset}{-1.05cm}
% \addtolength{\textheight}{2.1cm}

%\newcommand{\ale}[1]{{\bfseries [Ale:\ #1]}}
%\newcommand{\mm}[1]{{\bfseries [Mathias:\ #1]}}

\title{A Propagator for Maximum Weight String Matching with Arbitrary Pairwise Dependencies}

\author{Alessandro Dal Pal{\`u}$^1$, Mathias M{\"o}hl$^2$, and Sebastian Will$^{2,3}$}

\institute{Dipartimento di Matematica, Universit\`a� degli Studi di
  Parma, Parma, Italy, \email{alessandro.dalpalu@unipr.it} \and
  Bioinformatics, Institute of Computer Science,
  Albert-Ludwigs-Universit\"{a}t, Freiburg, Germany,
  \email{\{mmohl,will\}@informatik.uni-freiburg.de} \and Computation and
  Biology Lab, CSAIL, MIT, Cambridge MA, USA,
  \email{swill@csail.mit.edu}}


\begin{document}
\maketitle

\begin{abstract}
  The optimization of weighted string matchings is a well studied
  problem recurring in a number of application domains and can be
  solved efficiently. The problem becomes MAX-SNP-hard as soon as
  arbitrary pairwise dependencies among the matching edges are
  introduced.  We present a global propagator for this problem which
  is based on efficiently solving a relaxation of it.  In the context
  of bioinformatics, the problem is known as alignment of
  arc-annotated sequences, which is e.g. used for comparing RNA
  molecules. For a restricted version of this alignment problem, we
  show that a constraint program based on our propagator is on par
  with state of the art methods. For the general problem with
  unrestricted dependencies, our tool constitutes the first available
  method with promising applications in this field.
\end{abstract}

% WHAT TO SOLVE
% what is a weighted ordered partial matching?
The maximum weight string matching problem for strings $S_a$ and $S_b$
asks for a partial matching of positions in $S_a$ and $S_b$ that
preserves the string order and has maximum weight. This problem is
efficiently solved by dynamic programming
(DP)~\cite{smith81:_compar_bioseq}. An extended variant of the problem
introduces pairwise dependencies among positions in each string $S_a$
and $S_b$. In this problem variant, one optimizes the sum of weights,
where a weight is associated with each pair of matched positions and
with each pair of matched dependencies.  In general, this problem is
MAX-SNP-hard~\cite{blin07:_exten_hardn_rna_secon_struc_compar}.

% WHY TO SOLVE
% applications in Bioinformatics? "Arc-annotated sequence"

In bioinformatics, the problem has been studied as alignment of
arc-annotated sequences, where each arc represents a
dependency. There, the problem has applications in aligning RNA or
protein molecules that can be abstracted as sequences of monomers and
structural dependencies among those.

% what are arbitrary binary dependencies?
% - restricted dependencies (nested, restricted crossing) --> efficient

Due to the hardness of the problem, restricted versions with limited
dependencies, in particular nested and crossing ones, have been
considered. For nested and also certain restricted crossing
dependencies the alignment problem can be solved efficiently with DP
algorithms~\cite{Jiang:Lin:Ma:Gener_Edit_Dista:JCB2002,Moehl:Will:Backofen:PKalign:JCB2010}.
% - arbitrary crossing or unlimited --> MAX-SNP --> CP-application interesting
%
Heuristic approaches based on Integer Linear Programming (ILP) are
available for crossing dependencies specialized to RNA
~\cite{Bauer:Klau:Reinert:Accur_multi_seque:2007} and for unlimited
dependencies specialized to proteins~\cite{caprara02:_struc}.

\paragraph{Contribution} In this work, we consider a constraint
programming approach to the arc-annotated sequence alignment problem
with unlimited dependencies.  Our main contribution is a general
propagator for the maximum weight string matching with arbitrary
pairwise dependencies. The propagator is based on a relaxation that
resolves the dependencies by bounding their weight contribution. It
propagates on the upper bound of the total weight and prunes the valid
matchings accordingly. Furthermore, we discuss decomposition into
independent subproblems for optimization approaches using our
propagator.

% HOW TO SOLVE
% Propagator

% RESULTS/EVALUATION
% comparable to ILP,
% new applications possible (Riboswitch alignment)

% CONTRIBUTIONS
% - propagator
% - problem specific symmetry breaking technique in optimization problem
% - specialization and application for RNA alignment
We apply the technique to RNA sequence-structure alignment and show
that results are comparable to the state-of-the-art ILP approach
Lara~\cite{Bauer:Klau:Reinert:Accur_multi_seque:2007}. Finally, we
explore a new method for the comparison of riboswitches that could not
be handled by previous approaches for RNA, because they score at most
crossing dependencies.


\section{Preliminaries}
An \emph{arc-annotated sequence} is a pair $(S,P)$, where the
\emph{sequence} $S$ is a string over some alphabet $\Sigma$ and $P$ is
a set of \emph{arcs} $(i,j)$ with $1 \leq i< j \leq |S|$. We denote
the $i$-th symbol of $S$ by $S[i]$ and $S[i..j]$ is the subsequence
$S_iS_{i+1}\dots S_j$.

We distinguish crossing and unlimited sets of arcs.  A set $P$, where
each sequence position is involved in at most one arc, i.e. $\forall
(i,j)\neq(i',j')\in P: i\neq i' \land j\neq j' \land i\neq j \land
i'\neq j'$, is called \emph{crossing}. Otherwise it is called
\emph{unlimited}.

%%%%%%%%%%%%%%%%%%%%%%%%% ALIGNMENT (OF ENTIRE SEQUENCES)
An \emph{alignment $A$} of two arc-annotated sequences $(S_a,P_a)$ and
$(S_b,P_b)$ is a ordered partial matching between the positions of
$S_a$ and $S_b$. More precisely, $A\subseteq \{1,\dots,|S_a|\} \times
\{1,\dots,|S_b|\}$ has to satisfy for all
$(i,j),(i',j')\in A$ that 1.) $i>i'$ implies $j>j'$ and 2.)  $i=i'$
if and only if $j=j'$.
%
We define the \emph{$(i,i')$-prefix of $A$} as
$A\cap\PSET{(j,j')}{j\leq i,j'\leq i'}$ and the \emph{$(i,i')$-suffix
  of $A$} as $A\cap\PSET{(j,j')}{j>i,j'>i'}$.

%% Define alignment problem, i.p. scoring for unlimited structure.
Fix two arc-annotated sequences $(S_a,P_a)$ and $(S_b,P_b)$ with
unlimited structures $P_a$ and $P_b$. Define the \emph{weight of alignment $A$
of $(S_a,P_a)$ and $(S_b,P_b)$} as
\begin{displaymath}
  \weight(A) :=\!\!
  \sum_{(i,i')\in A} \!\!\sigma(i,i')
  \,+ \sum_{\substack{(i,j)\in P_a,(i',j')\in P_b,\\(i,i')\in A,(j,j')\in A}} \!\!\tau(i,j,i',j')\!\!
  \quad + \quad\gamma (|S_a|+|S_b|-2|A|),
\end{displaymath}
where $\sigma(i,i')$ is the similarity of sequence positions $S_a[i]$
and $S_b[i']$, $\tau(i,j,i',j')$ is the similarity of arcs $(i,j)\in
P_a$ and $(i',j')\in P_b$ and $\gamma$ is the gap cost associated with
each sequence position that is not matched.
% For $i'<i$, define $\tau(i',i,j',j):=\tau(i,i',j,j')$.

The \emph{alignment problem} is to determine
\begin{displaymath}
  \displaystyle\argmax_{\text{$A$ alignment of $(S_a,P_a)$ and $(S_b,P_b)$}} \weight(A).
\end{displaymath}
%
Note that on crossing arc annotation, the ILP approach
Lara~\cite{Bauer:Klau:Reinert:Accur_multi_seque:2007} solves
essentially the same problem. On unlimited input, Lara scores only a
crossing subset of the matched dependencies whereas our approach
scores all matches of dependencies.

\section{Constraint Model}

We model an alignment of arc-annotated sequences $(S_a,P_a)$ and
$(S_b,P_b)$ by variables $\var{MD}_i$ and $\var{M}_i$ for $1\leq i\leq
|S_a|$ with initial domains $D(\var{MD}_i) = \{1,\dots,|S_b|\}$ and
$D(\var{M}_i)=\{0,1\}$. We write $\vec{MD}$ and $\vec{M}$ to
denote the vectors of respective variables $\var{MD}_i$ and
$\var{M}_i$.

A valuation $V$ of these variables corresponds to a class $\cA(V)$ of
alignments $A$ of $(S_a,P_a)$ and $(S_b,P_b)$ as defined by
\begin{align*}
  V(\var{MD}_i)=j \land V(\var{M}_i)=1 &\text{ iff } (i,j) \in A  \\
  V(\var{MD}_i)=j \land V(\var{M}_i)=0 &\text{ iff } \nexists j \text{ with } (i,j) \in A \\&\quad\land \forall (i',j')\in A: i'< i \rightarrow j'\leq j \land i'>i \rightarrow j'>j.
\end{align*}

In this way, $M_i$ tells whether $i$ is matched or deleted and the
value $j$ of $MD_i$ tells that $i$ is matched to $j$ or deleted after
$j$. One can show that $\cA(V)$ has at most one element and that for
each alignment $A$ of $(S_a,P_a)$ and $(S_b,P_b)$ there is a
corresponding valuation.

For example, the following alignment $A=\{(1,1),(2,4),(4,5)\}$ of
$S_a=ACUG$ and $S_b=ACACG$, which is often written as
$\begin{array}{c}
  \texttt{A--CUG}\\
  \texttt{ACAC-G}
\end{array},$
corresponds to the valuation $\vec{MD}=(1,4,4,5)$ and $\vec{M}=(1,1,0,1)$.

We introduce a constraint \textsf{StringMatching}$(\vec{MD},\vec{M})$ that
is satisfied by any valuation with a corresponding alignment.
Furthermore, we model the weight of the alignment. Therefore, we
introduce a variable $\var{Weight}$ and a constraint
\textsf{StringMatchingWeight}$(\vec{MD},\vec{M},\var{Weight})$. This
constraint relates a valuation of $\var{MD}$ and $\var{M}$ to the
weight of its corresponding alignment.

Both constraints are propagated by our propagator described in the
next section. For finding optimal alignments we perform a
branch-and-bound search enumerating $\var{MD}$ and $\var{M}$ according
to a specific search strategy described after introducing the
propagator itself.

\section{The Alignment Propagator}

Our propagator computes hyper-arc consistency for
\textsf{StringMatching}$(\vec{MD},\vec{M})$ and propagates
\textsf{StringMatchingWeight}$(\vec{MD},\vec{M},\var{Weight})$.

It prunes $\vec{MD}$ and $\vec{M}$ due to the weight by computing
upper bounds of weights for single variable assignments and
furthermore computes lower and upper bounds for $\var{Weight}$ based
on $\vec{MD}$ and $\vec{M}$.
%
% Successfully applying branch-and-bound requires
% good upper bounds for the alignment weight, such that large parts of
% the search tree can be pruned.
Computing such bounds efficiently is essential for branch-and-bound
optimization.

Define the class $\cA(D)$ as union of $\cA(V)$ over all valuations $V$
that satisfy $D$. The computation of bounds is based on a relaxation of
the alignment problem. In this relaxation the two ends of each arc match are decoupled. Thus in the \emph{relaxed optimization
  problem for $D$}, we maximize a relaxed weight
 \begin{displaymath}
   \weight^{n,m}_\text{relaxed}(A) :=
   \sum_{(i,i')\in A} \left[
     \sigma(i,i') + \operatorname{ub}_D(i,i') \right]
  \quad + \quad \gamma (n+m-|A|),
\end{displaymath}
\vspace{-8pt}
over all alignments in $\cA(D)$, where $n=|S_a|$ and $m=|S_b|$ and
\begin{displaymath}
  \operatorname{ub}_D(i,i') :=\!\frac{1}{2} \max_{A\in\cA(D)} \sum_{\substack{(i,j)\in P_a,(i',j')\in P_b,\\(i,i')\in A,(j,j')\in A}} \!\!\!\tau(i,j,i',j')\,\,\,
   + \!\!\!\sum_{\substack{(j,i)\in P_a,(j',i')\in P_b,\\(i,i')\in A,(j,j')\in A}} \!\!\!\tau(j,i,j',i').
\end{displaymath}

Here, $\operatorname{ub}_D$ works as an upper bound for the weight
contributions by arc matches involving $(i,i')$ and consequently
$\weight^{|S_a|,|S_b|}_\text{relaxed}(A) \geq \weight(A)$ for $A\in\cA(D)$. Thus,
solving the relaxed problem yields an upper bound of \var{Weight}.

For a moment, postpone how to efficiently compute
$\operatorname{ub}_D(i,i')$. Then, because the relaxed weight has
the form of a sequence similarity score, one can apply the
Smith-Waterman algorithm~\cite{smith81:_compar_bioseq} to maximize the
relaxed weight in $O(n^2)$ by dynamic programming, where
$n=\max(|S_a|,|S_b|)$. The optimization problem is easily constrained
due to domain $D$, because domains directly restrict the valid cases
in the dynamic programming recursion.

Tracing back through the dynamic programming matrix yields an
alignment $A^l$. If $A_l$ also satisfies all other constraints of the
constraint problem, then $\weight(A^l)$ is a lower bound of $\var{Weight}$
if $A_l$. For the later studied RNA alignment problem, this bound can
always be propagated, since there are no other constraints.
%
Furthermore, we compute upper bounds for each single variable
valuation. This requires to complement the above ``forward algorithm''
that computes the matrix entries
\vspace{-4pt}
\begin{displaymath}
  \operatorname{Prefix}{(i,i')} := \max_{\text{$(i,i')$-prefix
      $A^p_{ii'}$ of }A\in\cA(D)} \weight^{i,i'}_\text{relaxed}(A^p_{ii'})
\end{displaymath}
by a symmetric ``backward algorithm'' that computes the matrix entries
\begin{displaymath}
 \operatorname{Suffix}{(i,i')} :=
  \max_{\text{$(i,i')$-suffix $A^s_{ii'}$ of }A\in\cA(D)}
  \weight^{|S_a|-i,|S_b|-i'}_\text{relaxed}(A^s_{ii'}).
\end{displaymath}

Now the variables $\vec{MD}$ can be pruned efficiently, because
$\operatorname{Prefix}{(i,i')}+\operatorname{Suffix}{(i,i')}$ is an
upper bound for the assignment $\var{MD}_i=j$. Similarly, we prune
$\vec{M}$ using the two matrices.


It remains to describe the efficient computation of
$\operatorname{ub}_D(i,i')$. It suffices to describe the maximization
of $\sum_{\substack{(i,j)\in P_a,(i',j')\in P_b,\\(i,i')\in
    A,(j,j')\in A}} \tau(i,j,i',j')$ over alignments in $\cA(D)$. A
single match $(j,j')$ can occur in an alignment in $\cA(D)$ if $j'\in
D(\var{MD}_j)$ and $1\in D(\var{M}_j)$. However, we look for the best
set of simultaneously valid matches $(j,j')$. The structure of this
subproblem is analogous to sequence alignment.
% DETAILS of upper bound by dynamic programming
For solving it efficiently, we define sorted lists
$j_1,\dots,j_l$ and $j'_1,\dots,j'_{l'}$ such that $(j_h,i)\in P_a$
for all $1\leq h\leq l$ and $(j'_{h'},i')\in P_b$ for all $1\leq
h'\leq l'$. We apply dynamic programming for evaluating
\begin{align*}
  \operatorname{UL}(0,0) &= 0 \quad
  \operatorname{UL}(h,0) = 0 \quad \operatorname{UL}(0,h') = 0\\
  \operatorname{UL}(h,h') &= \\
  &\hspace{-24pt}\max
  \begin{cases}
    UL(h-1,h') & \text{unless $\var{MD}_{j_h}=j'_{h'}$ and $\var{M}_{j_h}=1$}\\
    UL(h,h'-1) & \text{unless $\var{MD}_{j_h}=j'_{h'}$ and $\var{M}_{j_h}=1$}\\
    UL(h-1,h'-1) + \tau(j_h,j'_{h'},i,i') & \text{$j'_{h'} \in D(\var{MD}_{j_h})$ and $1\in D(\var{M}_{j_h})$}
  \end{cases}
\end{align*}
for $1\leq h\leq l$, $1\leq h'\leq l'$. Then, we perform the same
construction for the respective $r$ and $r'$ many arcs to the right of
$i$ and $i'$ and evaluate the corresponding recursion equation
$UR(h,h')$ for $1\leq h\leq r$, $1\leq h'\leq r'$.  Then,
$\operatorname{ub}_D(i,i') =
\operatorname{UL}(l,l')+\operatorname{UL}(r,r')$.

%% maybe give recursion equation? add details later
Therefore, $\operatorname{ub}_D(i,i')$ is computed in $O(ll'+rr')$ time.
For crossing dependencies $l+r=l'+r'=1$ and for many other applications $l+r$
and $l'+r'$ can be constantly
bounded~\cite{Will:etal:_infer_non_codin_rna_famil:PLOS2007} such that
the propagator runs in $O(n^2)$ time and space.

\paragraph{Affine Gap Cost}
In bioinformatics, penalizing unmatched positions using an affine
weighting function yields more realistic results. Our method is
straightforwardly extended to such a weighting scheme by using a
Gotoh-like forward and backward algorithm~\cite{Gotoh:82} in the
propagator without increasing its complexity.  It appears that this
modification comes more natural in our approach than the corresponding
extension in ILP, because it does not require any change of the model.

\paragraph{Propagator-guided Search Strategy}
To maximize the use of the propagator in a branch-and-bound setting,
we suggest a search strategy that aims at disproving overestimated
bounds fast and finding valid good alignments quickly. To achieve
this, information computed by the propagator in each propagation step
can be used to guide the search. In particular, this allows to select
a variable that yields a high undecided contribution to the upper
bound of the total weight. Furthermore, the computed backtrace
provides a good candidate for a solution, which can be favored by the
search strategy.

In our application to RNA, we select a variable with highest undecided
contribution to the upper bound and domain size as tie breaking. Its
domain is split such that the 20\% highest relaxed weights are chosen
first.

\section{Problem Decomposition}

%% string matching suggests a special kind of decomposition into subproblems along the string order (i.e. substrings)
Certain constraint optimization problems can be solved faster by
detecting independent subproblems during search and optimizing these
subproblems independently of each other. In general, this introduces
overhead for detecting dependencies and even interferes with
Branch-and-Bound when subproblems cannot be bounded well (confer
AND/OR search~\cite{Dechter:and_or_bab:AI}, which however doesn't
discuss decomposition in the presence of global propagators). However,
the string matching problem suggests a special form of decomposition
along the string order, where our propagator provides upper bounds for
the partial problems.
%% matching edges resolve the dependency due to "preserving string order"
The dependency due to the string order between the two subproblems for
variables $\var{MD}_1,\dots,\var{MD}_{i-1}$, $\var{M}_1,\dots,\var{M}_{i-1}$ and  $\var{MD}_{i+1},\dots,\var{MD}_{|S_a|}$, $\var{M}_{i+1},\dots,\var{M}_{|S_a|}$ is resolved as soon as a
matching edge $(i,j)$ is assigned (i.e. $\var{MD}_i=j$ and $\var{M}_i=1$).
%% remaining dependencies: distinguish internal (due to the arcs) and external.
The problem can be decomposed, if further the dependencies due
to arcs are resolved and the corresponding variables do not have
dependencies by constraints other than \textsf{StringMatching} and
\textsf{StringMatchingWeight}.
%% arc, where one end is fixed does not introduce dependency
Notably, for resolving the dependencies due to an arc $(i,j)$ it
suffices that one of its ends is matched, i.e. $\forall A\in D(\cA):
(i,i')\in A$ (or $\forall A\in D(\cA): (j,j')\in A$). Then one can
move the weight of each arc match $(i,j;i',j')$ to the match weight of
the other end $(j,j')$ (or $(i,i')$ respectively) and discard the dependency.
%% Gecode: no easy/efficient way to test for external dependency => implement only internal

In the RNA alignment problem, all dependencies can be checked within
the propagator.\footnote{This is important for our implementation,
  because general dependency checking for variables is expensive in
  Gecode.}  To avoid overhead, we apply the decomposition only if an
independent subproblem for $\var{MD}_{i},\dots,\var{MD}_{j}$,
$\var{M}_{i},\dots,\var{M}_{j}$ can be solved to optimality by the
propagator alone. This is the case, if all arc dependencies in the
subproblem are resolved. Then, the problem reduces to maximum weighted
string matching without dependencies and the traceback alignment
represents an optimal solution.  Assigning the traceback of the
subproblem to $\var{MD}_{i},\dots,\var{MD}_{j}$,
$\var{M}_{i},\dots,\var{M}_{j}$ is a form of symmetry breaking,
because it discards alignments with less or equal weight that are
systematically generated from other alignments in $\cA(D)$.


\section{Results}
The propagator and its application to RNA sequence-structure
alignment, called Carna, is implemented in C++ using the constraint
programming system Gecode. For handling input and output as well as
for special datastructures we reused code of
LocARNA~\cite{Will:etal:_infer_non_codin_rna_famil:PLOS2007}.

We run tests for two application scenarios. All experiments were
performed under 32-bit Linux on a T400s notebook with Intel P9600
CPU. First, we explore Carna's behavior on crossing input structure
using instances from all 16 Rfam families with crossing
structure. Table~\ref{tab:results} compares our results to
Lara~\cite{Bauer:Klau:Reinert:Accur_multi_seque:2007}. The table omits
all 8 instances where both approaches run in less than 0.1 seconds. In
all but one of the omitted cases, Carna solves the problem without
backtracking. In terms of performance, with the single exception of
tmRNA, both programs are on a par for the simpler class of crossing
structures.


\begin{table}
  \centering
  \begin{tabular}{c@{\hspace{6pt}}|@{\hspace{6pt}}c@{\hspace{6pt}}|@{\hspace{6pt}}c@{\hspace{6pt}}|@{\hspace{6pt}}c@{\hspace{6pt}}|@{\hspace{6pt}}c@{\hspace{6pt}}|@{\hspace{6pt}}c@{\hspace{6pt}}|@{\hspace{6pt}}c@{\hspace{6pt}}|@{\hspace{6pt}}c@{\hspace{6pt}}}
    Family
    & \multicolumn{2}{@{\hspace{-6pt}}|@{\hspace{-3pt}}c@{\hspace{3pt}}}{Lengths}
    & \multicolumn{2}{@{\hspace{-6pt}}|@{\hspace{-3pt}}c@{\hspace{3pt}}}{Run-time (s)}
    & \multicolumn{3}{@{\hspace{-6pt}}|@{\hspace{-3pt}}c}{Carna Search Tree} \\
    \hline\hline
                     &$S_a$&$S_b$& Carna& Lara & Depth & Fails & Size\\ %% user time
    \hline
    Entero\_OriR     & 126 & 130 & 0.03 & 0.18 & 38 & 13 & 50 \\
    Intron\_gpI      & 443 & 436 & 0.1  & 0.2  & 0  &  0 &  1 \\
    IRES\_Cripavirus & 202 & 199 & 0.2  & 0.04 &157 &127 &296 \\
    RNaseP\_arch     & 303 & 367 & 0.46 & 1.4  & 63 &  8 & 64 \\
    RNaseP\_bact\_b  & 408 & 401 & 3.0  & 2.3  &370 &677 &1463\\
    RNaseP\_nuc      & 317 & 346 & 0.07 & 2.9  & 14 &  4 & 16 \\
    Telomerase-vert  & 448 & 451 & 0.47 & 2.3  &146 & 32 &161 \\
    tmRNA            & 384 & 367 & 63   & 3.7  &433 &14347&28785\\
  \end{tabular}\\[6pt]
  \caption{Results for the eight harder instances of the benchmark set with crossing structures.
    We omit details for 8 instances where both programs run in less than 0.1 seconds. All times are given as user times.
  }
  \label{tab:results}
\end{table}

%% generated by svn r82
%% on T400s:  Intel(R) Core(TM)2 Duo CPU     P9600  @ 2.53GHz
%% : for i in Data/*.pk.fa ; do echo -n "$i "; echo `Utils/cRNAalign.pl $i 2>&1 | egrep '(Nodes|Fail|Depth|Memory|real|user|sys)'`; done

%%%% Data/Alpha_RBS.pk.fa Nodes: 90 Fail: 36 Depth: 54 Memory: 332296 real 0.06 user 0.06 sys 0.00
%%%% Data/Antizyme_FSE.pk.fa Nodes: 1 Fail: 0 Depth: 0 Memory: 19436 real 0.01 user 0.00 sys 0.00
%%%% Data/Corona_FSE.pk.fa Nodes: 25 Fail: 3 Depth: 24 Memory: 108928 real 0.01 user 0.01 sys 0.00
%%%% Data/Corona_pk3.pk.fa Nodes: 17 Fail: 2 Depth: 16 Memory: 51108 real 0.01 user 0.01 sys 0.00
% Data/Entero_OriR.pk.fa Nodes: 50 Fail: 13 Depth: 38 Memory: 265712 real 0.04 user 0.04 sys 0.00
% Data/Intron_gpI.pk.fa Nodes: 1 Fail: 0 Depth: 0 Memory: 134124 real 0.12 user 0.10 sys 0.01
% Data/IRES_Cripavirus.pk.fa Nodes: 296 Fail: 127 Depth: 157 Memory: 1717512 real 0.29 user 0.17 sys 0.12
%%%% Data/Parecho_CRE.pk.fa Nodes: 46 Fail: 6 Depth: 45 Memory: 233140 real 0.03 user 0.03 sys 0.00
%%%% Data/Prion_pknot.pk.fa Nodes: 6 Fail: 1 Depth: 5 Memory: 30056 real 0.01 user 0.00 sys 0.00
% Data/RNaseP_arch.pk.fa Nodes: 64 Fail: 8 Depth: 63 Memory: 1151660 real 0.60 user 0.46 sys 0.13
% Data/RNaseP_bact_b.pk.fa Nodes: 1463 Fail: 677 Depth: 370 Memory: 6646572 real 5.45 user 3.06 sys 2.40
% Data/RNaseP_nuc.pk.fa Nodes: 16 Fail: 4 Depth: 14 Memory: 264256 real 0.13 user 0.10 sys 0.03
%%%% Data/RydC.pk.fa Nodes: 1 Fail: 0 Depth: 0 Memory: 19436 real 0.01 user 0.00 sys 0.00
% Data/Telomerase-vert.pk.fa Nodes: 161 Fail: 32 Depth: 146 Memory: 2895672 real 0.87 user 0.47 sys 0.40
% Data/tmRNA.pk.fa Nodes: 28785 Fail: 14347 Depth: 433 Memory: 6848712 real 120.58 user 62.96 sys 57.61
%%%% Data/Tymo_tRNA-like.pk.fa Nodes: 51 Fail: 7 Depth: 50 Memory: 208944 real 0.02 user 0.01 sys 0.00


In our second scenario, we evaluate the behavior on the general class
of unlimited structures. Therefore, we apply the approach to the
alignment of riboswitches, which are RNA molecules with more than one
evolutionary conserved structure.  We annotate the RNA sequences by
the set of all base pairs with sufficiently high probability in the
RNA's structure ensemble~\cite{McCaskill:1990}. This set approximates
the overlay of the different riboswitch structures. In consequence,
the alignment is optimized with respect to all these structures
simultaneously. Since the set of base pairs is unlimited this
application has not been possible for existing approaches, which at
most score crossing structure. To benchmark the approach, we align 100
random instances from each out of 10 Rfam RNA families, which are
annotated as Riboswitch and confirmed by literature. We set a time
limit of 1 minute for each instance. The results are given in
Table~\ref{tab:riboswitches}. For this new problem, not all instances
could be solved within our strict time limit. However, the results
show that the approach handles all Riboswitch families sufficiently
well for bioinformatics applications with the only limitations for the
very longest sequences.

\begin{table}
  \centering
  \begin{tabular}{l|r|r|r|r|r}
% RF00059 & TPP  & TPP riboswitch THI element  \\
% RF00162 & SAM & SAM riboswitch S box leader \\
% RF00167 & Purine & Purine riboswitch \\
% RF00168 & Lysine & Lysine riboswitch \\
% RF00174 & Cobalamin &Cobalamin riboswitch \\
% RF00504 & Glycine   & Glycine riboswitch \\
% RF00521 & SAM\_alpha & SAM riboswitch alpha proteobacteria \\
% RF00634 & SAM-IV & S adenosyl methionine SAM riboswitch \\
% RF01055 & MOCO\_RNA\_motif & Moco molybdenum cofactor riboswitch \\
% RF01057 & SAH\_riboswitch & S adenosyl L homocysteine riboswitch \\
                %& \multicolumn{2}{c|}{Median Instance Size} & Median & Maximum & Exceeded\\
Family          & Length & Base pairs & Time (s) & Memory (MB) & Limit\\
\hline
SAH\_riboswitch &  79 &  81 & 0.13 &  3.3 &  2\% \\
SAM\_alpha 	&  79 &  96 & 0.03 &  1.9 &  0\% \\
Purine 	        & 101 &  74 & 0.07 &  2.3 &  0\% \\
Glycine         & 101 &  83 & 0.44 &  5.0 &  3\% \\
SAM 	        & 106 &  74 & 0.06 &  6.0 &  0\% \\
TPP     	& 107 &  96 & 0.43 &  8.7 &  13\% \\
SAM-IV 	        & 116 & 128 & 0.05 &  3.7 &  2\% \\
MOCO\_RNA\_motif& 141 & 111 & 0.24 &  9.4 &  10\% \\
Lysine 	        & 181 & 210 & 60 & 14.5 &  60\% \\
Cobalamin 	& 204 & 237 & 60 & 18.7 &  71\% \\
%
  \end{tabular}
  \caption{%
    Rfam Riboswitches. For 100 instances from each Rfam family annotated
    as riboswitch and confirmed by literature: Medians of average sequence
    length, average number of base pairs, and run-time (user time), maximal memory
    requirement and percentage of instances not solved to optimality
    within a given time limit of 1 min.
  }
  \label{tab:riboswitches}
\end{table}



\section{Discussion}

We presented a propagator for the problem of weighted string matching
with arbitrary dependencies, which is also known as the alignment
problem for arc-annotated sequences with unlimited structure.  Whereas
the problem itself is MAX-SNP-hard, our propagator makes it tractable
in a constraint programming framework by efficiently solving a
relaxation of the problem. Futhermore, we proposed a search strategy
that improves the benefit due to the propagator. Finally, we showed
that the weighted string matching problem can be decomposed into
independent subproblems during search. This allows for a AND/OR-type
optimization in the context of our global propagator.

To evaluate the applicability of our method in practice, we apply it
to the alignment of RNA structures. While all previous approaches in
this area are limited to score at most crossing subsets of the input
structures, our approach is able to align unlimited structures. This
is useful to align Riboswitches and other molecules with more than one
conserved structure, because it allows considering all their potential
structural conformations simultaneously.


\paragraph{Acknowledgments} This work is partially supported by DFG
grants WI 3628/1-1 and BA 2168/3-1 and PRIN-2008 (\emph{Innovative multi-disciplinary approaches
for constraint and preference reasoning}).

\bibliographystyle{plain}
\bibliography{all,refs}

\end{document}
