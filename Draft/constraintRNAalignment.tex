\documentclass[11pt]{llncs}

\usepackage{amsmath}

\newcommand{\SET}[1]{\{#1\}} 
\newcommand*{\PSET}[2]{\{\,#1\mid#2\,\}}

\newcommand{\score}{\operatorname{score}}
\newcommand{\argmax}{\operatorname*{arg max}}

\title{Alignnment of RNA with Structures of Unlimited Complexity}

\author{Alessandro Dal Pal{\'u}$^1$, Mathias M{\"o}hl$^2$, and Sebastian Will$^{2,3}$}

\institute{Dipartimento di Matematica, Università degli Studi di
  Parma, Parma, Italy, \email{alessandro.dalpalu@unipr.it} \and
  Bioinformatics, Institute of Computer Science,
  Albert-Ludwigs-Universit\"{a}t, Freiburg, Germany,
  \email{\{mmohl,will\}@informatik.uni-freiburg.de} \and Computational
  Biology Lab, CSAIL, MIT, Cambridge MA, USA,
  \email{swill@csail.mit.edu}}

\begin{document}
\maketitle

\begin{abstract}
  Sequence-structure alignment of RNAs with arbitrary secondary
  structures is MAXSNP-hard. Therefore, the problem of RNA alignment
  is commonly restricted to nested structure, where dynamic
  programming yields efficient solutions. However, nested structure
  cannot model pseudoknots or even more complex structural
  dependencies.  Nevertheless those dependencies are essential and
  conserved features of many RNAs. Only few existing approaches deal
  with crossings of limited complexity or arbitrary crossing
  structures. Here, we present a constraint approach for alignment of
  structures in the even more general class of structures with
  unlimited complexity. Our central contribution is a new RNA
  alignment propagator. It is based on an efficient $O(n^2)$
  relaxation of the RNA alignment problem. Specifically our approach
  solves the alignment problem for sequences with given input
  structure of unlimited complexity. The method is implemented using
  Gecode.
\end{abstract}

In the post-genomic era, biologists get more and more interested in
studying non-coding RNA molecules with catalytic and regulatory
activity as central players in biological systems. 
%
The computational analysis of non-coding RNA requires to take
strucural information into account.
% is more involved than the analysis of DNA
%and proteins, because many non-coding RNAs, like proteins, form
%structure that is essential for their function and is consequently
%evolutionary conserved. However, unlike for proteins, it is much more
%common for homologuous non-coding RNAs that their similarity is barely
%visible at the sequence level but almost exclusively in their
%structure.  
%
Whereas RNAs form three-dimensional strucures, structural analysis of
RNA is usually concerned with the secondary structure of an RNA, i.e.
the set of RNA base pairs $(i,j)$ that form contacts between the bases
$i$ and $j$ via H-bonds. The RNA alignment problem is to align two RNA
sequences $A$ and $B$ together with given secondary structure for each
RNA such that a score based on sequence and structure similarity is
optimized. The difficulty of this problem depends on the complexity of
the RNA structures. Therefore, a complexity hierarchy of RNA
structures was introduced. Most RNA analysis is done for the class of
nested structures $P$, where base-pairs do not cross, because for this
class one can find efficient dynamic programming algorithms for
structure prediction and alignment for reasonable scoring
schemes~\cite{Zuker,Jiang}. The class of crossing RNA structures $P$
restricts the degree of base pairing to at most one, as it is commonly
assumed for single RNA structure. Prediction and alignment in this
class is NP-hard in general~\cite{...}. However, one can devise a
number of algorithms that efficiently predict or align RNAs with
structures from classes in between non-crossing and arbitrary
crossing.~\cite{PK,structure,prediction,Moehl:Lifting}. However these
algorithms have complexities that limit their application range. A
different approach is to handle nested structures with parametrized
complexity, where the parameter describes the complexity of the
structure(s), an example is~\cite{Moehl:CPM} for alignment.  Finally,
the ILP approach Lara aligns arbitrarily complex crossing structures
and turns out to be more effective than the dynamic programming based
approaches. The success of this AI technique was a strong motivation
for this work, where we study the alignment of RNAs with structures of
unlimited comlexity using constraint programming.
\paragraph{Contribution} We devise a constraint algorithm for the
problem of aligning two RNA molecules with given sequences and
arbitrary, i.e. unlimited secondary structures. We describe the
constraint model, develop a special RNA alignment propagator, and
present a specific search strategy. The approach was implemented using
the Gecode constraint programming system. Finally, we give preliminary
results of our approach.

\section{Methods}

\subsection{Preliminaries}
An \emph{RNA sequence} $S$ is a string over the set of bases
$\{A,U,C,G\}$ and an \emph{RNA structure} $P$ is a set of base pairs
$(i,j)$ with $1 \leq i< j \leq |S|$. Define an \emph{arc-annotated
  sequence} as pair of RNA sequence and RNA structure.  We denote the
$i$-th symbol of $S$ by $S[i]$.

One can devise a hierarchy of RNA structure classes based on the
following properties. Two arcs $(i,j)$ and $(i','j)$ are called
\emph{nested} iff $i<i'<j'<j$ or $i'<i<j<j'$, they are
\emph{independent} iff $i<j<i'<j'$ or $i'<j'<'<j$.

A RNA structure $P$ is called \emph{nested} if all differing base
pairs $(i,j),(i',j')\in P$ are either nested or independent.  In a
crossing RNA structure $P$ each base is involved in at most one base
pair, i.e. $\forall (i,j)\neq(i',j')\in P: i\neq i' \land j\neq j'
\land i\neq j \land i'\neq j'$. We use the term \emph{unlimited} to
refer to an arbitrary RNA structure. Note that by definition each
nested structure is crossing, and each crossing structure is
unlimited.

%%%%%%%%%%%%%%%%%%%%%%%%% ALIGNMENT (OF ENTIRE SEQUENCES)
An \emph{alignment $A$} of two \emph{arc-annotated sequences}
$(S_a,P_a)$ and $(S_b,P_b)$ is a set $A_m\cup A_g$, where
$A_m\subseteq [1..|S_a|] \times [1..|S_b|]$ is a set of \emph{match
  edges} such that for all $(i,j),(i',j')\in A_m$ holds 1.) $i>i'$
implies $j>j'$ and 2.)  $i=i'$ if and only if $j=j'$ and $A_g$ is the
set of \emph{gap edges} $\PSET{(x,-)}{x\in [1..|S_a|]\land \not\exists
  y: (x,y)\in A_m}\cup\PSET{(-,y)}{y\in [1..|S_b|]\land \not\exists x:
  (x,y)\in A_m}$.
%
We define the \emph{$(i,i')$-prefix of $A$} as
$A\cap(\PSET{(j,j')}{j\leq i,j'\leq i'}\cup\PSET{(j,-)|j\leq i}\cup\PSET{(-,j')|j'\leq i'}$
and the  \emph{$(i,i')$-suffix of $A$} as
$A\cap(\PSET{(j,j')}{j>i,j'>i'}\cup\PSET{(j,-)|j>i}\cup\PSET{(-,j')|j'>i'}$.


%% Define alignment problem, i.p. scoring for unlimited structure.
Fix two arc-annotated sequences $(S_a,P_a)$ and $(S_b,P_b)$ with for
unlimited structures $P_a$ and $P_b$. Define the score of an
alignment A of $(S_a,P_a)$ and $(S_b,P_b)$ as
\begin{displaymath}
  \score(A_m\cup A_g) :=
  \sum_{(i,i')\in A_m} \sigma(i,i')
  + \sum_{\substack{(i,j)\in P_a,(i',j')\in P_b,\\(i,i')\in A_m,(j,j')\in A_m}} \sigma(i,j,i',j')
  + \gamma |A_g|,
\end{displaymath}
where $\sigma(i,j)$ is the similarity of bases $S_a[i]$ and $S_b[j]$
and $\tau(i,j,i',j')$ is the similarity of base pairs $(i,j)\in P_a$
and $(i',j')\in P_b$.
% For $i'<i$, define $\tau(i',i,j',j):=\tau(i,i',j,j')$.

The \emph{alignment problem} is to determine 
\begin{displaymath}
  \argmax_{\text{$A$ alignment of $(S_a,P_a)$ and $(S_b,P_b)$}} \score(A).
\end{displaymath}


\newcommand{\var}{\texttt}
\newcommand{\cA}{{\mathcal A}}
\subsection{Constraint Model}

We model an alignment of arc-annotated sequences $(S_a,P_a)$ and
$(S_b,P_b)$ by variables $\var{MD}_i$ and $\var{M}_i$ for $1\leq i\leq
|S_a|$ with initial domains $D(\var{MD}_i) = \{1,\dots,|S_b|\}$ and
$D(\var{M}_i)=\{0,1\}$. We write $\vec{MD}$ and $\vec{M}$ to
denote the vectors of respective variables $\var{MD}_i$ and
$\var{M}_i$.

A valuation $V$ of these variables corresponds to a class $\cA(V)$ of
alignments $A$ of $(S_a,P_a)$ and $(S_b,P_b)$ as defined by
\begin{align*}
  V(\var{MD}_i)=j \land V(\var{M}_i)=1 &\text{ iff } (i,j) \in A  \\
  V(\var{MD}_i)=j \land V(\var{M}_i)=0 &\text{ iff } (i,-) \in A \\&\quad\land \forall (i',j')\in A: i'< i \rightarrow j'\leq j \land i'>i \rightarrow j'>j.
\end{align*}

One can show that $\cA(V)$ has size of at most one and that for each
alignment A there is a corresponding valuation.

For example, the following alignment and valuation correspond to each
other: $A=\{(1,1),(-,2),(-,3),(2,4),(3,-),(4,5)\}$, which is often written as
$\begin{array}{c}
\texttt{A--CUG}\\
\texttt{ACAC-G}
\end{array},$
corresponds to the valuation $\vec{MD}=(1,4,4,5)$ and $\vec{M}=(1,1,0,1)$.

Notably, alignments corresponding to a valuation that assigns $M_i=j$
can be composed from an alignment of prefixes $S_a[1..i]$ and
$S_b[1..j]$ and an alignment of suffixes $S_a[i+1..|S_a|]$ and $S_b[j+1..|S_b|]$
regardless of $\var{M}_i$.

We introduce a constraint \texttt{Alignment}$(\vec{MD},\vec{M})$ that is
satisfied by any valuation with a corresponding alignment.  

In addition to modelling alignments, we need to model the score of the
alignment. Therefore, we introduce a variable $\var{Score}$ together
with a constraint
\texttt{AlignmentScore}$(\vec{MD},\vec{M},\var{Score})$. This constraint
relates a valuation of $\var{MD}$ and $\var{M}$ to the score of its
corresponding alignment.

Both constraints are propagated by the propagator of the next
subsection. For finding optimal alignments we perform a
branch-and-bound search enumerating $\var{MD}$ and $\var{M}$ according
to a specific search strategy described at the end of the next
section. Applying branch-and-bound succesfully requires good upper
bounds for the alignment score, such that large parts of the search
tree can be pruned. Computing such a bound efficiently is the central
job of the alignment propagator.

\subsection{The Alignment Propagator}

The alignment propagator computes hyper-arc consistency for constraint
\texttt{Alignment}$(\vec{MD},\vec{M})$ and propagates
\texttt{AlignmentScore}$(\vec{MD},\vec{M},\var{Score})$.

It prunes $\vec{MD}$ and $\vec{M}$ due to the score by computing upper
score bounds for single variable assignments and furthermore computes
a lower and upper bounds for $\var{Score}$ based on $\vec{MD}$ and
$\vec{M}$.

Define the class $\cA(D)$ as union of $\cA(V)$ for all valuations that
satisfy D. The computation of bounds is based on a relaxation of the
alignment problem. In this relaxation the two ends of each base pair
match are decoupled. Thus in the \emph{relaxed optimization problem
  for $D$}, we maximize a relaxed score
 \begin{displaymath}
  \score_\text{relaxed}(A_m\cup A_g) :=
  \sum_{(i,i')\in A_m} 
  \sigma(i,i') + 0.5\operatorname{ubound}_D(i,i')
  + \gamma |A_g|,
\end{displaymath}
over all alignments in $\cA(D)$, where
\begin{align*}
  \operatorname{ubound}_D(i,i') := & \max_{A\in\cA(D)} \sum_{\substack{(i,j)\in P_a,\\(i',j')\in P_b,\\(i,i')\in A,\\(j,j')\in A}} \tau(i,i',j,j')
  & + \max_{A\in\cA(D)} \sum_{\substack{(j,i)\in P_a,\\(j',i')\in P_b,\\(i,i')\in A,\\(j,j')\in A}} \tau(i',i,j',j).
\end{align*}

It is easy to see that $\score_\text{relaxed}(A) \geq \score(A)$ for
$A\in\cA(D)$. Thus, solving the relaxed problem yields an upper bound
of \var{Score}. 

For a moment, postpone how to efficiently compute
$\operatorname{ubound}_D(i,i')$. Then, because the relaxed score has
the form of a sequence similarity score, one can apply the
Smith-Waterman algorithm~\cite{SmithWaterman} to maximize the relaxed
score in $O(n^2)$ by dynamic programming. The optimization problem is
easily constrained due to domain $D$, because domains directly
restrict the valid cases in the dynamic programming recursion.

Tracing back through the dynamic programming matrix yields an alignment
$A^l$ such that $\score{A^l}$ is a lower bound of $\var{Score}$.

Furthermore, we compute upper bounds for each single variable
valuation. This requires to complement the above ``forward algorithm''
that computes the matrix
\begin{align*}
  \operatorname{Prefix}{(i,i')} := & \max_{\text{$(i,i')$-prefix
      $A^p_{ii'}$ of }A\in\cA(D)} \score_\text{relaxed}(A^p_{ii'})
  \intertext{by a symmetric ``backward algorithm'' computing the
    matrix entries} \operatorname{Suffix}{(i,i')} := &
  \max_{\text{$(i,i')$-suffix $A^s_{ii'}$ of }A\in\cA(D)}
  \score_\text{relaxed}(A^s_{ii'}).
\end{align*}

Now the variables $\vec{MD}$ can be pruned efficiently, because
$\operatorname{Prefix}{(i,i')}+\operatorname{Suffix}{(i,i')}$ is an
upper bound for the assignment $\var{MD}_i=j$. Similarly, we prune
$\vec{M}$ using the two matrices.

It remains to describe the efficient computation of
$\operatorname{ubound}_D(i,i')$. It suffices to describe the
maximization of $\sum_{\substack{(i,j)\in P_a,(i',j')\in
    P_b,\\(i,i')\in A,(j,j')\in A}} \tau(i,i',j,j')$ over alignments
in $\cA(D)$. A match (j,j') is valid in any alignment in $\cA(D)$ if
and only of $j'\in D(\var{MD}_j)$ and $1\in D(\var{M}_j)$. However, we
look for the best set of simultaneously valid matches $(j,j')$. This
subproblem has a similar struture to sequence alignment and is thus
solved efficiently by dynamic programming and
$\operatorname{ubound}(i,i')$ is computed in $O(kk')$ time, where $k$
and $k'$ are the respective number of base pairs incident to $i$ and
$i'$. For many applications $k$ and $k'$ can be constanly bounded
such that the propagator runs in $O(n^2)$ time and space.

\paragraph{Propagator-guided search strategy}
%Idea of guiding the search by backtrace.
Our search strategy guides the search to disprove overestimated bounds
fast and to find valid good alignments quickly. Because information
for achieving both goals is computed during propagation and is
expensive to recompute, we reuse propagation results for guiding the
search. We select a variable with large domain size and yields a
high undecided contribution to the upper bound. We split the domain of
this variable to select the 20\% highest relaxed scores first.

%\subsection{Visualization of partial solutions}

%\subsection{Implementation details}
%Use of LocARNA code.

%Hacks that were applied to make things work in Gecode.

\section{Results}

The method is implemented in C++ using the constraint programming
system Gecode. 

{\bf Here we need to add some more concrete results. We can write about the following:}
We run preliminary tests for several application scenarios. First, we
try the behavior with crossing input structure, where our alignment
problem is comparable to the problems solved by PKalign, ... and Lara
on fixed input structures. Our prototype solves alignments from all
Rfam families with crossing structure in very fast time and was able
to align RNaseP and tmRNA. Compared to this, our dynamic programming
for alignment of pseudoknotted RNA is already much slower on the less
complex instances of the test set.  The performance of Lara on this
benchmark set is comparable to our approach.

Our second application scenario is aligning dot-plot matrices. RNA
dot-plots as computed by RNAfold\cite{RNAfold} contain probabilities
for each base pair. By filtering with a probability cut-off, we obtain
an unlimited structure that contains only interesting base pairs. We
align such unlimited structures using base pair similarities that are
derived from the base pair probabilities. This approach is known from
PMcomp or LocARNA. The result is an alignment that is not only guided
by a single common structure but the common structural potential of
both RNAs.

\section{Discussion}

We could demonstrate that a constraint-based approach to RNA alignment
could be competitive with the ILP based method Lara for crossing
structures. Moreover, the approach is the first such method for
unlimited structure. The full potential of this method and potential
biological applications, e.g. for aligning dot-plots have still to be
explored. Having a constraint-based method promises flexibility for
further extensions and improvements. Solving relaxed problems in
propagators for handling crossing and unlimited RNA structure was
shown to be a viable approach and appears to be generalizable to
related problems.

\end{document}
